\begin{table}
    \centering

    \begin{threeparttable}
        \caption{Variáveis utilizadas nos modelos econométricos}

        \begin{tabular}{m{0.18\linewidth}|m{0.7\linewidth}}
            Variável & Descrição \\
            \toprule\toprule
            Tratamento & Variável binária que assume 1 caso o município tenha recebido passe livre e 0 caso contrário \\
            \midrule[0.2pt]
            Competitividade & Quão proximos foram os resultados entre primeiro e segundo candidatos. Calculado por $1/[\log({V_A/V_B})]$, na qual $V_A$ representa o número de votos recebidos pelo candidato mais votado e $V_B$, pelo segundo candidato mais votado.\footnotemark \\
            \midrule[0.2pt]
            População & População do município \\
            \midrule[0.2pt]
            PIB per capita & PIB per capita do municipio, disponível até 2020. Para 2022, foram utilizados os últimos dados disponíveis, de 2020. \\
            \midrule[0.2pt]
            Beneficiados & Número de pessoas no municipio dividido pela quantidade de eleitores aptos. \\
            \midrule[0.2pt]
            IDEB & Nota da educacao dos anos finais do ensino fundamental nas escolas publicas. A nota é apenas calculadas nos anos ímpares, com o primeiro dado disponível em 2005, então foram utilizados dados defasados em um ano. \\
            \midrule[0.2pt]
            PIB governo & É definido como valor adicionado bruto a preços correntes da administração, defesa, educação e saúde públicas e seguridade social dividido pelo PIB municipal. \\
            \midrule[0.2pt]
            Eleitores por seção & Média do número de eleitores aptos por seção eleitoral no município \\

            \label{tab_variaveis}
        \end{tabular}
        \vspace{10pt}
        \begin{tablenotes}
            \small
            \item $^{\thefootnote}$ Esta variável é calculada com base em dados póstumos à votação, mas a premissa adotada para incluir esta variável é de que as pesquisas eleitorais e o sentimento dos votantes em relação à competitividade não são significativamente diferentes dos resultados observados \textit{a posteriori}
            
        \end{tablenotes}
    \end{threeparttable}
    
\end{table}
