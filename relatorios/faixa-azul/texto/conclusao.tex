\section{Conclusão}

Diante da análise empírica realizada sobre os sinistros de trânsito na Avenida dos Bandeirantes, em São Paulo, após a implementação da faixa azul, algumas conclusões podem ser destacadas. As evidências apontam que a faixa azul causou uma redução no número de sinistros envolvendo pessoas feridas. Desde a sua inauguração, em outubro de 2022 até dezembro de 2023\footnote{Os dados estão disponíveis até este instante, consultado em 18/01/2024. A divulgação dos dados é defasada em alguns meses}, estima-se que foram evitados 32 sinistros que envolveriam motociclistas e ao menos uma pessoa ferida de forma grave ou leve, o que representa uma redução de 26.8\% dos sinistros registrados no período pós tratamento.

Apesar de demonstrar significância estatística, a análise apresenta uma série de limitações. Primeiramente e mais importante, foi identificado um efeito local, que não necessariamente pode ser extrapolado para outras avenidas. Em segundo, não foi estimado o custo de implementação da faixa azul ou uma análise de custo benefício dessa política pública. Além disso, pelos poucos períodos pós-tratamento disponíveis, a longevidade do efeito não foi estimada, sendo possível que o efeito da faixa azul se dissipe ou se fortaleça ao longo dos próximos meses. Por fim, não foi feita uma análise de transbordamentos da medida. 

Seria interessante avaliar qual o impacto da faixa azul sobre o fluxo de veículos, bem como sua velocidade -- especialmente para o caso das motocicletas. Possivelmente, a medida pode ter incentivado o uso de mais motocicletas, o que vai na contramão do que se propõe em cidades modernas de incentivar o transporte público coletivo, que é mais seguro e possui melhores externalidades.

Em suma, o estudo apresenta evidências que indicam o sucesso da faixa azul na Avenida dos Bandeirantes segundo o critério de redução do número de sinistros envolvendo vítimas com ferimentos leves ou graves. Entretanto, para inferir sobre os benefícios da faixa azul em escala municipal, estadual ou federal, é necessário analisar mais avenidas que receberam o tratamento. Diversas avenidas estão recebendo a implementação da faixa azul\footnote{A primeira faixa azul foi implementada na Avenida 23 de Maio, em janeiro de 2022; seguida da Avenida dos Bandeirantes, em outubro de 2022; Avenida Prestes Maia em outubro de 2023; Avenidas Sumaré, Paulo VI, Miguel Yunes e Av. das Nações Unidas em novembro de 2023; Avenidas Brigadeiro Faria Lima, Luiz Dumont Villares, Zaki Narchi, Jacu Pêssego e do Estado em dezembro de 2023.} e na medida em que os dados mais recentes se tornarem disponíveis, serão viáveis estudos para analisar o impacto da medida.

